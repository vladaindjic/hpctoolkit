%% $Id$

%%%%%%%%%%%%%%%%%%%%%%%%%%%%%%%%%%%%%%%%%%%%%%%%%%%%%%%%%%%%%%%%%%%%%%%%%%%%%
%%%%%%%%%%%%%%%%%%%%%%%%%%%%%%%%%%%%%%%%%%%%%%%%%%%%%%%%%%%%%%%%%%%%%%%%%%%%%

\documentclass[english]{article}
\usepackage[latin1]{inputenc}
\usepackage{babel}
\usepackage{verbatim}

%% do we have the `hyperref package?
\IfFileExists{hyperref.sty}{
   \usepackage[bookmarksopen,bookmarksnumbered]{hyperref}
}{}

%% do we have the `fancyhdr' or `fancyheadings' package?
\IfFileExists{fancyhdr.sty}{
\usepackage[fancyhdr]{latex2man}
}{
\IfFileExists{fancyheadings.sty}{
\usepackage[fancy]{latex2man}
}{
\usepackage[nofancy]{latex2man}
\message{no fancyhdr or fancyheadings package present, discard it}
}}

%% do we have the `rcsinfo' package?
\IfFileExists{rcsinfo.sty}{
\usepackage[nofancy]{rcsinfo}
\rcsInfo $Id$
\setDate{\rcsInfoLongDate}
}{
\setDate{ 2018/08/27}
\message{package rcsinfo not present, discard it}
}

\setVersionWord{Version:}  %%% that's the default, no need to set it.
\setVersion{=PACKAGE_VERSION=}

%%%%%%%%%%%%%%%%%%%%%%%%%%%%%%%%%%%%%%%%%%%%%%%%%%%%%%%%%%%%%%%%%%%%%%%%%%%%%
%%%%%%%%%%%%%%%%%%%%%%%%%%%%%%%%%%%%%%%%%%%%%%%%%%%%%%%%%%%%%%%%%%%%%%%%%%%%%

\begin{document}

\begin{Name}{1}{hpcproftt}{The HPCToolkit Performance Tools}{The HPCToolkit Performance Tools}{hpcproftt:\\ Correlation of Flat Profile Metrics for Teletype Output}


\Prog{hpcproftt} generates textual dumps of call path profiles recorded by hpcrun.

\end{Name}

%%%%%%%%%%%%%%%%%%%%%%%%%%%%%%%%%%%%%%%%%%%%%%%%%%%%%%%%%%%%%%%%%%
\section{Synopsis}

\Prog{hpcproftt} \Arg{profile-file} ...\\
\Prog{hpcproftt} \Arg{-V}\\
\Prog{hpcproftt} \Arg{-h}


%%%%%%%%%%%%%%%%%%%%%%%%%%%%%%%%%%%%%%%%%%%%%%%%%%%%%%%%%%%%%%%%%%
\section{Description}

\Prog{hpcproftt} generates textual dumps of call path profiles recorded
by hpcrun.  The profile list may contain one or more call path profiles.

This tool principally intended for use by HPCToolkit developers rather
than end users. Under some circumstances, a user might be interested in
examining the raw measurement data (1) if HPCToolkit's tool chain has a problem processing
the measurement data, or (2) if he or she is interested in inspecting
measurements data associated with an individual instruction, which is below the
line-level attribution reported by HPCToolkit's graphical user interfaces.

The first section of the output describes the profile header. The profile
header contains information including the name of the executable, its
path, and the value of the PATH environment variable when the program was
run, along with other metadata including the batch scheduler job id, MPI
rank, thread ID, host ID, process ID on the host, and times when the first
and last samples were recorded if tracing was enabled during an execution.

The second section describes the epoch header. Conceptually, measurement
of an execution is divided into one or more epochs in time. Measurements
taken during an epoch are collected and flushed to disk separately. This
section is mostly of historical interest as now hpcrun typically collects
measurements in a single epoch.

The third section of the profile reports the metric table.  This lists
all metrics collected during the execution of an MPI rank or thread.
Information about a metric includes its name, type, format, and whether
there is another partner metric.  The show and showPercent fields
indicate whether the metrics should be displayed by default in hpcviewer.
The period field is used to scale the number of samples collected, e.g. to
convert samples collected every 1M cache misses into cache miss counts.

The next section of the output reports information that HPCToolkit's
collected about the application's load map during execution.  The load
map consists of a series of pairs, where an executable or shared
library is assigned an integer load module ID, known as an lm-id.
Call path samples consist of a sequence of (load module ID, offset)
pairs. Each pair represents a location in an application binary or one
of its associated shared libraries. Load module IDs present in a call
path profile generally correspond to load module IDs in this table. In
certain cases, the special dummy load module ID 0 used to represent CCT
nodes that are not associated with object code. The calling context tree
root is an example of one of these nodes.

The last component of the file is the Calling Context Tree (CCT). Each
node in the tree has an ID, the ID of its parent in the tree, an
(lm-id, lm-ip) pair that represents the load module and a relative
instruction pointer within that load module that represents a location
in a call path.  A node in the tree may or may not have non-zero metric
values associated with it. If a node's ID is negative, that means the
node is a leaf in the tree. If a node's ID is odd, that means that it
was recorded in an associated trace file and must be handled carefully
during post-processing.

%%%%%%%%%%%%%%%%%%%%%%%%%%%%%%%%%%%%%%%%%%%%%%%%%%%%%%%%%%%%%%%%%%
\section{Arguments}

\begin{Description}
\item[\Arg{profile-file} ...] A list of one or more call path profiles.
\end{Description}

\subsection{Options}

\begin{Description}
\item[\Opt{-V}, \Opt{--version}] Print version information.
\item[\Opt{-h}, \Opt{--help}] Print help.
\end{Description}

%%%%%%%%%%%%%%%%%%%%%%%%%%%%%%%%%%%%%%%%%%%%%%%%%%%%%%%%%%%%%%%%%%
%\section{Notes}

%%%%%%%%%%%%%%%%%%%%%%%%%%%%%%%%%%%%%%%%%%%%%%%%%%%%%%%%%%%%%%%%%%
\section{See Also}

\HTMLhref{hpctoolkit.html}{\Cmd{hpctoolkit}{1}}.

%%%%%%%%%%%%%%%%%%%%%%%%%%%%%%%%%%%%%%%%%%%%%%%%%%%%%%%%%%%%%%%%%%
\section{Version}

Version: \Version

%%%%%%%%%%%%%%%%%%%%%%%%%%%%%%%%%%%%%%%%%%%%%%%%%%%%%%%%%%%%%%%%%%
\section{License and Copyright}

\begin{description}
\item[Copyright] \copyright\ 2002-2020, Rice University.
\item[License] See \File{README.License}.
\end{description}

%%%%%%%%%%%%%%%%%%%%%%%%%%%%%%%%%%%%%%%%%%%%%%%%%%%%%%%%%%%%%%%%%%
\section{Authors}

\noindent
Rice University's HPCToolkit Research Group \\
Email: \Email{hpctoolkit-forum =at= rice.edu} \\
WWW: \URL{http://hpctoolkit.org}.

\LatexManEnd

\end{document}

%% Local Variables:
%% eval: (add-hook 'write-file-hooks 'time-stamp)
%% time-stamp-start: "setDate{ "
%% time-stamp-format: "%:y/%02m/%02d"
%% time-stamp-end: "}\n"
%% time-stamp-line-limit: 50
%% End:

